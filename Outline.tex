\documentclass[11pt, oneside, usenames, dvipsnames, svgnames, table, final]{amsart}

\usepackage[T1]{fontenc}
\usepackage[utf8]{inputenc}

\newcommand{\Inn}{\operatorname{Inn}}
\newcommand{\Out}{\operatorname{Out}}
\newcommand{\Isom}{\operatorname{Isom}}

%% Old-fashioned appearance

% \usepackage{baskervillef}
% \usepackage[varqu,varl,var0]{inconsolata}
% \usepackage[scale=.95,type1]{cabin}
% \usepackage[varbb]{newtxmath}
% \usepackage[cal=boondoxo]{mathalfa}


%% \usepackage{fourier}

%% \usepackage[lmdh]{lmodern}

%% \usepackage{mathptmx}

\usepackage{charter}
\frenchspacing


\usepackage{tbjwHeader}


\usepackage[backref=true,style=alphabetic,url=false,citestyle=alphabetic]{biblatex}
\addbibresource{BenW_Standard_BibTeX5.bib}
\addbibresource{Article.bib}

\newcommand{\benw}[2][]{\ifdraft{\todo[linecolor=Green,backgroundcolor=Green!25,bordercolor=Green,#1]{#2---Ben W.}}{}}
\newcommand{\secondauth}[2][]{\ifdraft{\todo[linecolor=Red,backgroundcolor=Red!25,bordercolor=Red,#1]{#2---2nd Auth.}}{}}
\newcommand{\thirdauth}[2][]{\todo[linecolor=Blue,backgroundcolor=Blue!25,bordercolor=Blue,#1]{#2}}

\begin{document}   

\title{Character varieties, bundles of algebras and symmetries}
\author{Ben Williams} %\thanks{Edit this}
\address{Department of Mathematics, University of British Columbia, Vancouver~BC V6T~1Z2, Canada}
\email{tbjw@math.ubc.ca}

% \author{E. Talia}

% \date{Friday March 7, 2014.}
%\subjclass{Primary ??? Secondary ???. }


% \begin{abstract}

% \end{abstract}
\maketitle

% \tableofcontents


\section{Representations} \label{sec:Introduction}

The theory of representations is described at length in many textbooks. Particularly recommended are \cite[Ch.~1,2 and 6]{Serre1977} or \cite{Harris2013}. Most sources concentrate on the case of finite groups, which is not the case of interest to us. For the specific case of linear representations of fundamental groups of $3$-manifolds, the monograph \cite{Shalen2002} is useful.
\medskip

Throughout $\Gamma$ denotes a group, typically infinite but generated by finitely many elements and relations (finitely
presented).

Let $k$ denote a field, generally $\CC$, and let $n \ge 1$ be an integer. For now, $M=\Mat_{n \times n}(k)$ denotes the
$k$-algebra of $n\times n$ matrices. Within $M$, we can find the groups $\GL(n;k)$ and $\SL(n;k)$ of invertible matrices
and matrices with determinant $1$ respectively.

The \emph{group algebra} of $\Gamma$ is the $k$-algebra $k\Gamma$ where the elements are finite formal sums
$\sum_{\gamma_i \in \Gamma} a_i\gamma_i$, each $a_i$ being drawn from $k$. Addition of two formal sums is obvious (add
component by component) and multiplication is mostly obvious (it distributes over addition and $(a\gamma)(a' \gamma') =
aa' (\gamma \gamma')$ where we use the operation in $\Gamma$ to multiply $\gamma \gamma'$.)

A \emph{representation} of $\Gamma$ consists of either of the following:
\begin{enumerate}
\item A homomorphism of groups $\phi : \Gamma \to \GL(n;k)$;
\item A homomorphism of algebras $\phi: k\Gamma \to \Mat_{n \times n}(k)$.
\end{enumerate}
It is worth thinking through why the two definitions implicit above are equivalent.

One way of thinking of a representation is that $\phi$ defines an action of the abstract group $\Gamma$ on the
relatively concrete object $k^n$.

A representation $\phi$ is \emph{irreducible} if there is no positive-dimensional, proper subspace $V \subset k^n$ that
is invariant under the action of $\phi(\Gamma)$/ At the two extremes, $\{0\}$ and $k^n$ are certainly invariant under
the action of $\phi(\Gamma)$. A \emph{reducible} representation is one that is not irreducible.

If $\phi$ is a reducible representation, then we can find some $V \isom k^s$ for some $s \in \{1, \dots, n-1\}$ so that
$\phi(\Gamma)$ restricts to an action on $V$. This amounts to a \emph{subrepresentation} of $\phi$.

\begin{remark}
Let us assume for the moment that $k = \bar k$ is algebraically closed. Since we will generally work with $k = \CC$,
this is not a severe condition. Then a theorem of Burnside, \cite{Lomonosov2004}, says that a
representation $\phi : k \Gamma \to \Mat_{n \times n}(k)$ is irreducible if and only if it is surjective.

For a non-algebraically-closed field, the situation is a little more complicated: consider the $\Mat_{2\times 2}(\RR)$ representation of the
group $\Gamma=(\ZZ/(4), +)$ given by \[ \phi(x) =
\begin{bmatrix}
  \cos(\pi x/2) & -\sin(\pi x/2) \\ \sin(\pi x/2) & \cos(\pi x/2)
\end{bmatrix}.\] This representation is irreducible since any invariant proper subspace of $\RR^2$ would have to consist
of an eigenspace of $\phi(x)$ for all $x$, but for $x=1$, the matrix $\phi(x)$ has no real eigenspaces. Nonetheless,
the map $\phi$, viewed as an algebra map $\phi : \RR \Gamma \to \Mat_{2 \times 2}(\RR)$ is not surjective: the image
is a $2$-dimensional subalgebra of the $4$-dimensional space $\Mat_{2 \times 2}(\RR)$.
\end{remark}

We say two representations $\phi, \psi : \Gamma \to \GL(n;k)$ are \emph{equivalent} (or \emph{isomorphic}) if they
differ by a change of coordinates. That is, if there exists some $P \in \GL(n; k)$ so that $P^{-1} \phi(g) P = \psi(g)$
for all $g \in \Gamma$. The order of quantifiers is important here: the element $P$ does not vary with $g$, but is fixed
for all $g \in \Gamma$.

The choice of coordinates is often not fixed when we are discussing representations. Therefore the things we care about
are usually representations-up-to-equivalence, and so we should develop techniques for distinguishing between them. The
most important technique for distinguishing between inequivalent representations are characters.

\begin{definition}
  If $\phi : \Gamma \to \GL(n;k)$ is a representation, then the \emph{character} of $\phi$ is the composite function
  \[ \Tr \circ \phi : \Gamma \to k \]
  where $\Tr$ denotes the trace of a matrix.
\end{definition}

The trace of a matrix is invariant under change of coordinates, so that characters of representations are invariant
under change of coordinates: they are invariants of the isomorphism class of the representation.

For finite groups and the field $k=\CC$, the characters completely determine the isomorphism classes of representations.

\section{Special linear groups}
\label{sec:spec-line-groups}

It is possible to restrict the definition of representation by constraining the codomain group. That is, we might
consider only \emph{$\SL(n)$-representations}, which is a homomorphism $\phi: \Gamma \to \SL(n; k)$. For such
representations, the tight link to algebra homomorphisms is broken.

\section{Character varieties}
\label{sec:character-varieties}


Fix $k = \CC$ in this section, and a target group: either $\GL(n, \CC)$ or $\SL(n, \CC)$.

If $\Gamma$ is finitely presented at least, then it is possible to produce ``geometric'' objects the points of which
correspond to equivalence classes of representations of $\Gamma$ for a fixed target group. Such an object is called a
``character variety''. There are many variations on this idea, and we will not define the term ``character variety''
precisely.


\begin{example}
  For a finite group $\Gamma$ and a fixed $n$, there are only finitely many equivalence classes of representations
  $\phi: \Gamma \to \GL(n; \CC)$, and therefore the character varieties of finite groups consist of finitely many points
  and are not very interesting.
\end{example}

\begin{example}
  Let $\Gamma = \ZZ$, arguably the easiest infinite group to think about, and consider the target group $\GL(1; \CC) =
  \CC^\times$, the easiest target group. Here a homomorphism $\phi : \Gamma \to \GL(1; \CC)$ is completely determined by
  $\phi(1)$ and all values of $\phi(1)$ are possible. Since $\GL(1; \CC)$ is abelian, representations are isomorphic if
  and only if they are equal. Therefore the character variety in this case is the geometric object $\CC^\times$,
  parametrizing the possible values of $\phi(1)$.
\end{example}

\begin{example}
  Let $\Gamma = \ZZ$, and consider the target group $\SL(2; \CC)$, the next easiest target group. Here a homomorphism $\phi : \Gamma \to \SL(2; \CC)$ is completely determined by
  $\phi(1)=A$, which is $2\times 2$ matrix of determinant $1$. Up to conjugacy in $\SL(2; \CC)$, the matrix $A$ is
  usually completely determined by the unordered pair of its eigenvalues $\lambda, \lambda^{-1}$, and therefore $A$ is
  usually completely determined by its trace: $\lambda + \lambda^{-1}$. The exception to the ``usually'' is where
  $\lambda =  \lambda^{-1}$ (and both are $\pm 1$). In this case, the matrix $A$ need not be diagonalizable, and may be conjugate to either
  of
  \[
    \begin{bmatrix}
     \pm  1 & 1 \\ 0 &  \pm 1 
    \end{bmatrix}, \quad
    \begin{bmatrix}
     \pm 1 & 0 \\ 0 & \pm 1 
    \end{bmatrix}.
  \]
  All values of $\Tr(A) \in \CC$ are possible, and mostly determine the isomorphism class of the representation, but
  there is some peculiar behaviour at $2$ and $-2$.

  Note that in this example, all the representations that arise are reducible (every square complex matrix has an
  eigenvector), but the peculiar behaviour we saw at $2$ and $-2$ involved matrices $\pm I_2$ that were ``more reducible
  than usual''.
\end{example}



\begin{example}
  This example is drawn from \cite{Munoz2009}.
  
  Fix two natural numbers $n,m$ that are relatively prime. Let $\Gamma = \langle x,y \mid x^n = y^m \rangle$. This is
  the fundamental group of the complement of an $(m,n)$-torus knot.

  Again, let us consider the target group $\SL(2; \CC)$. A homomorphism $\rho: \Gamma \to \SL(2; \CC)$ consists of a
  pair of matrices $A=\rho(x)$ and $B=\rho(y)$ satisfying $A^n = B^m$. There are many reducible representations, and we
  will disregard them, preferring to consider only the character variety of irreducible representations.


  If the representation $\rho$ is irreducible, then $A$ and $B$ do not share any complex eigenspaces, but the equation
  $A^n=B^m$ requires $A^n$ to have precisely the same eigenspaces as $B^m$. The eigenspaces of $A^n$ agree with the
  eigenspaces of $A$ except possibly in the case where $A$ has two different eigenvalues $\alpha_1$ and $\alpha_2$ such
  that $\alpha_1^n = \alpha_2^n$. Since the representation in question has target $\SL(2; \CC)$, it must be the case
  here that $\alpha_1^{-1} = \alpha_2$, and so $\alpha_1^n = \pm 1$.

  An identical argument applies to $B$. We deduce that in any irreducible representation of $\Gamma$, the matrix $A$ is
  diagonalizable with eigenspaces $V_1$, $V_2$ and matching eigenvalues $\alpha_1$ and $\alpha_1^{-1}$ where $\alpha_1^n
  = 1$. Similarly, $B$ is diagonalizable with eigenspaces $W_1$, $W_2$ and matching eigenvalues $\beta_1$ and
  $\beta_1^{-1}$, where $\beta_1^m = \pm 1$.

  Since $A$ and $B$ themselves do not have eigenspaces in common, $\alpha \neq \pm 1$ and $\beta \neq \pm 1$.

  Conjugacy of the matrices amounts to moving the eigenspaces around. Therefore we can arrange for $V_1, V_2$ and $W_1$
  to be the $\CC$-linear span of
  \[
    \begin{bmatrix}
      1 \\ 0 
    \end{bmatrix},
    \begin{bmatrix}
      0 \\ 1 
    \end{bmatrix},
    \begin{bmatrix}
      1 \\ 1 
    \end{bmatrix}.
  \]
  One checks that indeed this is possible. Then $W_2$ is the span of some other vector
  \[
    \begin{bmatrix}
      1 \\ z
    \end{bmatrix}.
  \]
  where $z_2$ is not $0$ or $1$.

  In summary, from a conjugacy class of $A$ and $B$ we have isolated $\alpha_1$ and $\beta_1$ which are $n$-th and
  $m$-th roots of $\pm 1$, themselves different from $\pm 1$, and we have isolated $z \in \CC \sm \{0,1\}$. We made
  certain choices to arrive at the triple $(\alpha_1, \beta_1, z )$. We might have chosen $\alpha_1^{-1}$ instead, which
  is to say we  swap what we think of as the $x$-coordinate and which is the $y$-coordinate. This results in replacing
  $z$ by $1/z$. Similarly, we might have chosen $\beta_1^{-1}$ instead, in which case we should scale the $y$-coordinate
  by $1/z$, again leading to an exchange of $z$ with $1/z$.

  The irreducible $\SL(2;\CC)$-character variety of the $m,n$-torus knots consists of the space of equivalence-classes of
  triples
  \[ (\alpha_1, \beta_1, z) \quad \text{s.t.} \quad\alpha_1^n = \pm 1, \, \beta_1^m = \pm 1, \, \alpha \neq \pm, \, \beta
    \neq \pm 1, \, z \in \C \sm \{0,1\}, \]
  under equivalence relation
  \[ (\alpha_1, \beta_1, z)  \sim (\alpha_1^{-1}, \beta_1, z^{-1}) \sim (\alpha_1, \beta_1^{-1}, z^{-1})  \sim
    (\alpha_1^{-1}, \beta_1^{-1}, z).\]

  Note that we used a slightly different parameter from \cite{Munoz2009}, so the presentation we arrived at is
  different: our $z$ is their $r/(r-1)$.
\end{example}


\section{Outer automorphisms}
\label{sec:outer-automorphisms}

If $\Gamma$ is a group, then an \emph{automorphism} of a group is a homomorphism $\phi: \Gamma \to \Gamma$ that has an
inverse homomorphism, $\phi^{-1}$. It is an exercise in group theory to show that a homomorphism $\phi$ is an automorphism if and only
if it is bijective (i.e., the inverse function is automatically a homomorphism).

If $g \in \Gamma$ is an element, then conjugation by $g$ yields an automorphism of $\Gamma$:
\[ h \mapsto g^{-1} h g. \]
Such an automorphism is said to be an \emph{inner automorphism} of $\Gamma$ because it arises directly from $\Gamma$.

\begin{remark}
  A group $\Gamma$ is abelian if and only if every inner automorphism is trivial.
\end{remark}

The composite of two inner automorphisms (conjugation by $g_0$, conjugation by $g_1$) is again inner (conjugation by
$g_0 g_1$). Therefore the set of inner automorphisms, $\Inn(\Gamma)$, forms a subgroup of the group of automorphisms,
$\Aut(\Gamma)$.

\begin{exercise}
  The group $\Inn(\Gamma)$ is a normal subgroup of $\Aut(\Gamma)$.
\end{exercise}


An automorphism of $\Gamma$ that is not inner is called an \emph{outer automorphism}. In contrast to inner
automorphisms, the outer automorphisms do not obviously form a group (the identity automorphism is inner). On the other
hand, there is a quotient group
\[ \Out(\Gamma) : = \Aut(\Gamma)/\Inn(\Gamma) \]
called the \emph{outer automorphism group} of $\Gamma$. The elements of $\Out(\Gamma)$ are equivalence classes of
automorphisms of $\Gamma$. The class of the identity is exactly the group of inner automorphisms, and two automorphisms
$\xi$, $\eta$ lie in the same element of $\Out(\Gamma)$ if $\xi \circ \eta^{-1}$ is an inner automorphism.

\begin{exercise}
  Calculate $\Out(D_n)$ where $D_n$ denotes the dihedral group of symmetries of a regular $n$-gon.
\end{exercise}

\section{Action on Character Varieties}
\label{sec:acti-char-vari}

Suppose $\rho : \Gamma \to \GL(n,k)$ is a representation and $\phi$ is an automorphism of $\Gamma$. Then we can produce
a new representation of $\Gamma$ by precomposition with $\phi$:
\[ \rho \circ \phi : \Gamma \to \GL(n,k). \]
This endows the set of representations with a right $\Aut(\Gamma)$-action.

If $\rho_0$ and $\rho_1$ are equivalent representations, then $\rho_0 \circ \phi$ and $\rho_1 \circ \phi$ are also
equivalent, so that $\Aut(\Gamma)$ acts on any reasonably-defined character variety\footnote{Remember, we have not been
  specific about what constitutes a character variety.}

The next proposition is the main technical point of this document:
\begin{proposition}
  If $\rho_0, \rho_1 : \Gamma \to \GL(n,k)$ are equivalent representations and $\phi_0$, $\phi_1$ are automorphisms of
  $\Gamma$ so that $\phi_0^{-1} \phi_1$ is an inner automorphism, then $\rho_0 \circ \phi_0$ is equivalent to $\rho_1
  \circ \phi_1$.
\end{proposition}
\begin{proof}
  Let $g \in \GL(n,k)$ be such that $g^{-1} \rho_0(\gamma) g = \rho_1(\gamma)$ for all $\gamma \in \Gamma$. Let $\delta \in \Gamma$ be such that
  $\phi_0^{-1} \phi_1(\gamma) = \delta \gamma \delta^{-1}$ for all $\gamma \in \Gamma$. Then
  \[ \rho_1 \circ \phi_1(\gamma) = \rho_1(\phi_0(\delta) \phi_0(\gamma) \phi_0(\delta)^{-1}) = g^{-1}
    \rho_0(\phi_0(\delta)) \rho_0(\phi_0(\gamma)) \rho_0(\phi_0(\delta))^{-1} g \]
  so that conjugation by $\rho_0(\phi_0(\delta))^{-1} g$ takes $\rho_0 \circ \phi_0$ to $\rho_1 \circ \phi_1$, as required.
\end{proof}
\begin{remark}
  In spite of the mess of calculations, the above is a follow-your-nose proof. There is really only one thing to do at
  each stage.
\end{remark}

\begin{corollary}
  The action of $\Aut(\Gamma)$ on a character variety factors through an action of $\Out(\Gamma)$.
\end{corollary}


\begin{example}
  The torus knot group $\Gamma = \langle x,y \mid x^n = y^m \rangle$ has an automorphism $\sigma: \Gamma \to \Gamma$ given
  by $\sigma(x) =x^{-1}$ and $\sigma(y) = y^{-1}$. Note that $\sigma(xy)= x^{-1}y^{-1} \neq (xy)^{-1}$, so this is not
  the inversion anti-automorphism. The automorphism $\sigma$ represents the unique nontrivial outer automorphism class of
  $\Gamma$.


  Now we will see how this automorphism acts on the character variety. To be honest, we have not fully developed the
  character variety of torus knot groups, but we have determined the important subset corresponding to irreducible
  representations. Let $x$ be such a point, corresponding to the equivalence class of some homomorphism
  \[  \rho(x) = A = \begin{bmatrix} \alpha & 0 \\ 0 & \alpha^{-1} \end{bmatrix}, \quad \rho(y) = B =
    \frac{1}{z-1}\begin{bmatrix} \beta z - \beta^{-1} & \beta^{-1} - \beta \\ \beta z - \beta^{-1} z & \beta^{-1} z -
      \beta \end{bmatrix}. \]
  Here we have chosen coordinates so that the first matrix is diagonal and the second has eigenvectors $\begin{bmatrix}
    1 \\ 1 \end{bmatrix}$ with eigenvalue $\beta$ and $\begin{bmatrix} 1 \\ z \end{bmatrix}$ with eigenvalue
  $\beta^{-1}$.

  The involution of the group has the effect of replacing $x$ by $x^{-1}$ and so $A$ by $A^{-1}$, and so $\alpha$ by
  $\alpha^{-1}$, and it has the effect of replacing $y$ by $y^{-1}$ and so $B$ by $B^{-1}$, and so $\beta$ by
  $\beta^{-1}$. The parameter $z$ is left unchanged.

  That is, the effect of this automorphism is to replace the class of the triple $(\alpha, \beta, z)$ with the class of
  the triple $(\alpha^{-1}, \beta^{-1}, z)$. Since this is the same class, the overall effect of the automorphism on
  these points in the character variety is to do nothing at all.  
\end{example}

\begin{remark}
    This is left as a remark because it would take too long to prove all the details. There exists a class of 
\end{remark}


\section{Tautological bundles of algebras}
\label{sec:taut-bundl-algebr}

The title of this section is a fancy way of describing the following phenomenon.

Fix a source group $\Gamma$. For the time being let us take $G = \GL(n; k)$, but other options
are open to us.

Suppose $x \in X$ is a point in the character variety corresponding to an equivalence class of irreducible
representations $k \Gamma \to \Mat_{n \times n}(k)$. The lack of specificity inherent in the term ``equivalence class''
can be conceptualized in two different ways.
\begin{enumerate}
\item What we have done up to now is to consider fixed algebras $k \Gamma$ and $\Mat_{n \times n}(k)$ and an equivalence
  class of homomorphisms $\rho : k \Gamma \to \Mat_{n \times n}(k)$ between them.
\item Note that $\ker(\rho)$, a $2$-sided ideal of $k \Gamma$, does not change if we replace $\rho$ by a conjugate
  homomorphism. We can consider a fixed abstract algebra $A_x = k \Gamma / \ker(\rho)$ and then consider all the
  different isomorphisms of $k$-algebras $A_x \to \Mat_{n \times n}(k)$. We can factor any homomorphism $\rho$
  as the quotient $q_x : k \Gamma \to A_x$ followed by an isomorphism of $A_x$ with $\Mat_{n \times n}(k)$.
\end{enumerate}

\begin{remark}
  The story is a little more complicated when $x$ corresponds to a reducible class of representations: in that case
  $A_x \to \Mat_{n \times n}(k)$ is an injective morphism with predetermined image.
\end{remark}

Restricting to cases where the representations are irreducible, the algebras $A_x$ depend on $x$, and are abstractly
generated as algebras by the images $q_x(\gamma)$ where $\gamma$ ranges over a set of generators of  $\Gamma$. These
generators $q_x(\gamma)$ satisfy some relations making $A_x$ abstractly isomorphic to $\Mat_{n \times n}(k)$. It is
usually not worth the trouble to write the relations down.

The $A_x$ depend continuously on $x$, in a way we will not make precise but should be apparent in examples. Together,
they form a ``tautological bundle of algebras'' over the subset of the character variety corresponding to the
irreducible representations.



\section{Induced actions on the tautological algebra}







\printbibliography


 \end{document}

 
%%% Local Variables:
%%% mode: latex
%%% TeX-master: t
%%% End:
